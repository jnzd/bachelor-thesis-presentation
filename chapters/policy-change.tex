\section{Algorithm and Policy Change}


\begin{frame}
    \frametitle{Policy Change}
    \begin{itemize}
        \item Start a new monitor with the new policy
        \item \textbf{Goal:} Our monitor evaluates the new policy at the current time point just as if it had seen the same trace as the old monitor
        \item \textbf{Naive approach:} Read entire trace again
        \item \textbf{Idea:} Reduce the size of the trace by removing events that do not influence how the new policy gets evaluated
    \end{itemize}
\end{frame}

\begin{frame}{Optimization Example}
    \vspace{-1cm}
    \begin{align*}
        \phi = 
        (P(x, "a") \Since{[0,30)} Q(x, 2))
        \land 
        (\neg (R(y, "c", z)) \Since{[20,60)} P(y, z))
    \end{align*}

    \begin{itemize}
        \item \textbf{Relative Interval:} $\RI(\phi) = (-60,0]$
        \item \textbf{Extended Relative Intervals:} $\ERI(\phi) =$
        \begin{alignat*}{3}
            &P(*,*)      &&\to (-60,-20]\\
            &P(*,"a")    &&\to (-30,0]\\
            &Q(*,2)      &&\to (-30,0]\\
            &R(*,"c",*)  &&\to (-60,0]
        \end{alignat*}
        \item All time points with time stamps not in the interval
        % \vspace{-.2cm}
        % \begin{equation*}
            $\{\text{current time stamp}\} \oplus \RI(\phi)$
        % \end{equation*}
        % \vspace{-1.5cm}
        do not change the evaluation of $\phi$ at the current time point.
    \end{itemize}


\end{frame}

\subsection{Relative Intervals}
\begin{frame}{Interval Operators}
    Let $I$ and $J$ be two intervals, then
    \begin{itemize}
        \item $I \oplus J = \{i+j \mid i \in I, j \in J \}$  
        \begin{itemize}
            \item $[0,3] \oplus [-2,4] = [-2,7]$
        \end{itemize}
        \item $I \Cup J$ is the smallest interval that contains all elements that are in at least one of the intervals $I$ and $J$.
        \begin{itemize}
            \item $[-4,1] \Cup [4,5] = [-4,5]$
        \end{itemize}
    \end{itemize}
    
\end{frame}


\begin{frame}
    \frametitle{Relative Intervals}
    \begin{definition} 
        \label{def:rel-int}
        The relative interval of the formula $\phi$, $\RI(\phi) \subseteq \mathbb{Z}$ is defined inductively over the formula structure: 
        $\RI(\phi) =$
        \begin{equation*}
            \begin{cases}
                \{0\}     & \text{atomic formulas} \\ 
                \RI(\psi) & \text{$\neg \psi$, 
                                    $\exists x.\psi$,} \forall x.\psi \\
                \RI(\psi) \Cup \RI(\chi) & \text{$\psi \lor \chi, 
                                                \psi \land \chi$} \\
                (-b,0] \Cup ((-b,-a] \oplus \RI(\psi)) & \text{$\Previous{[a,b)}\psi$} \\
                [0,b) \Cup ([a,b) \oplus \RI(\psi)) & \text{$\Next{[a,b)}$}\\
                (-b,0] \Cup ((-b,0] \oplus \RI(\psi)) \Cup ((-b,-a] \oplus \RI(\chi)) & \text{$\psi \Since{}_{[a,b)} \chi$} \\
                [0,b) \Cup ([0,b) \oplus \RI(\psi)) \Cup ([a,b) \oplus \RI(\chi)) & \text{$\psi \Until{[a,b)} \chi$} \\
                % [0,b) \Cup ([0,b) \oplus \RIr(\rho)) & \text{$\Fregex{[a,b)} \rho$, and}\\
                % (-b,0] \Cup ((-b,0] \oplus \RIr(\rho)) & \text{$\Pregex{[a,b)} \rho$.}\\
            \end{cases}
        \end{equation*}
    \end{definition}
    {\footnotesize
    Basin et al. "Scalable Offline Monitoring of Temporal Specifications". In: \textit{Formal Methods ind System Design 49 (1 2016)}, pp. 75-108
    }
\end{frame}


\subsection{Extended Relative Intervals}
\begin{frame}
    \frametitle{Extended Relative Intervals}

    \begin{definition}
        \label{def:e-rel-int-ops}
        Let $M$ and $N$ be two masked predicate maps and $T$ a positive interval, then 
        \begin{align*}
            M \Cupmerge N = 
                & \{ p(l) \rightarrow (I \Cup J) \mid 
                    p(l) \rightarrow I \in m \text{ and } 
                    p(l) \rightarrow J \in n \} \\
                & \cup \{p(l) \rightarrow I \mid  
                    (p(l) \rightarrow I \in m \text{ and }
                    p(l) \in \keys(M) \setminus \keys(N)) \} \\
                & \cup \{p(l) \rightarrow I \mid  
                    (p(l) \rightarrow I \in n \text{ and }
                    p(l) \in \keys(N) \setminus \keys(M))
                    \} \\
            T \Cupext M = 
                & \{ p(l) \rightarrow (T \Cup I) \mid 
                    p(l) \rightarrow I \in M \} \\
            T \oplusext M = 
                & \{ p(l) \rightarrow (T \oplus I) \mid 
                    p(l) \rightarrow I \in M \} \\
        \end{align*}
    \end{definition}
    
\end{frame}


\begin{frame}
    \frametitle{Relative Intervals}
    \begin{definition} 
        \label{def:rel-int}
        The relative interval of the formula $\phi$, $\RI(\phi) \subseteq \mathbb{Z}$ is defined inductively over the formula structure: 
        $\RI(\phi) =$
        \begin{equation*}
            \begin{cases}
                \{0\}     & \text{atomic formulas} \\ 
                \RI(\psi) & \text{$\neg \psi$, 
                                    $\exists x.\psi$,} \forall x.\psi \\
                \RI(\psi) \Cup \RI(\chi) & \text{$\psi \lor \chi, 
                                                \psi \land \chi$} \\
                (-b,0] \Cup ((-b,-a] \oplus \RI(\psi)) & \text{$\Previous{[a,b)}\psi$} \\
                [0,b) \Cup ([a,b) \oplus \RI(\psi)) & \text{$\Next{[a,b)}$}\\
                (-b,0] \Cup ((-b,0] \oplus \RI(\psi)) \Cup ((-b,-a] \oplus \RI(\chi)) & \text{$\psi \Since{}_{[a,b)} \chi$} \\
                [0,b) \Cup ([0,b) \oplus \RI(\psi)) \Cup ([a,b) \oplus \RI(\chi)) & \text{$\psi \Until{[a,b)} \chi$} \\
                % [0,b) \Cup ([0,b) \oplus \RIr(\rho)) & \text{$\Fregex{[a,b)} \rho$, and}\\
                % (-b,0] \Cup ((-b,0] \oplus \RIr(\rho)) & \text{$\Pregex{[a,b)} \rho$.}\\
            \end{cases}
        \end{equation*}
    \end{definition}
    {\footnotesize
    Basin et al. "Scalable Offline Monitoring of Temporal Specifications". In: \textit{Formal Methods ind System Design 49 (1 2016)}, pp. 75-108
    }
\end{frame}

\begin{frame}
    \frametitle{Extended Relative Intervals}
    \Wider{
    $\ERI(\phi) =$
    \vspace{-.2cm}
    \begin{equation*}
        \begin{cases}
            \textcolor{red}{ \{\} } & \text{atomic formulas} \\
                                    & \text{excl. predicates} \\
            \textcolor{red}{\{p(m) \rightarrow [0,0]\}} 
                & \text{predicate } p\\
                & \text{with mask } m, \\
            \ERI(\psi) 
                & \neg \psi, \exists x.\psi, \forall x.\psi \\
            \ERI(\psi) \textcolor{red}{\Cupmerge} \ERI(\chi) 
                & \psi \lor \chi, \psi \land \chi \\
            (-b,0] \textcolor{red}{\Cupext} ((-b,-a] \textcolor{red}{\oplusext} \ERI(\psi)) 
                & \Previous{[a,b)} \psi \\
            [0,b) \textcolor{red}{\Cupext} ([a,b) \textcolor{red}{\oplusext} \ERI(\psi)) 
                & \Next{[a,b)} \psi \\
            (-b,0] \textcolor{red}{\Cupext} ((-b,0] \textcolor{red}{\oplusext} \ERI(\psi)) \textcolor{red}{\Cupmerge} ((-b,-a] \textcolor{red}{\oplusext} \ERI(\chi)) 
                & \psi \Since{[a,b)} \chi \\
            [0,b) \textcolor{red}{\Cupext} ([0,b) \textcolor{red}{\oplusext} \ERI(\psi)) \textcolor{red}{\Cupmerge} ([a,b) \textcolor{red}{\oplusext} \ERI(\chi)) 
                & \psi \Until{[a,b)} \chi \\
            % [0,b) \Cupext ([0,b) \oplusext \ERIr(\psi))
            %     & \Fregex{[a,b)} \psi, \text{ and}\\
            % (-b,0] \Cupext ((-b,0] \oplusext \ERIr(\psi)) 
            %     & \Pregex{[a,b)} \psi.
        \end{cases}
    \end{equation*}
    We proofed that this definition is a correct approximation such that it extracts all necessary time points from a trace and that the evaluation of $\phi$ is the same for the original trace and the extraction.
    }
    
\end{frame}

\begin{frame}{Extended Relative Intervals - SQL Query}
    \begin{align*}
        \phi = 
        (P(x, "a") \Since{[0,30)} Q(x, 2))
        \land 
        (\neg (R(y, "c", z)) \Since{[20,60)} P(y, z))
    \end{align*}
    \begin{itemize}
        \item \textbf{Relative Interval:} $\RI(\phi) = \textcolor{blue}{(-60,0]}$
        \item \textbf{Extended Relative Intervals:} $\ERI(\phi) =$
        \begin{alignat*}{3}
            &P(*,*)                       &&\to \textcolor{red}{(-60,-20]}\\
            &P(*,\textcolor{red}{"a"})    &&\to \textcolor{red}{(-30,0]}\\
            &Q(*,\textcolor{red}{2})      &&\to \textcolor{red}{(-30,0]}\\
            &R(*,\textcolor{red}{"c"},*)  &&\to \textcolor{red}{(-60,0]}
        \end{alignat*}
    \end{itemize}
\end{frame}

\begin{frame}[fragile]{Extended Relative Intervals - SQL Query}
\begin{alltt}
SELECT * FROM P WHERE
    \textcolor{red}{(time_stamp > <NOW>-60 AND time_stamp <= <NOW>-20)}
    OR
    \textcolor{red}{(x2 = "a" AND
    time_stamp > <NOW> -30 AND time_stamp <= <NOW>-0)};
SELECT * FROM Q WHERE
    \textcolor{red}{ (x2 = 2 AND
    time_stamp > <NOW>-30 AND time_stamp <= <NOW>-0)};
SELECT * FROM R WHERE
    \textcolor{red}{ (x2 = "c" AND
    time_stamp > <NOW>-60 AND time_stamp <= <NOW>-0)};
SELECT * FROM time_points WHERE
    \textcolor{blue}{(time_stamp > <NOW>-60 AND time_stamp <= <NOW>-0)};
\end{alltt}
\end{frame}


\begin{frame}{Database Result Conversion}

\textbf{QuestDB Response}
    \begin{alltt}
[{"R":[]},\\
{"Q":[[0,8,65536,"Thu, 01 Jan 1970 00:00:01 GMT"]]},\\
{"P":[]},\\
{"time\_points":[[65536,"Thu, 01 Jan 1970 00:00:01 GMT"],
                 [65537,"Thu, 01 Jan 1970 00:00:02 GMT"]]}]
    \end{alltt}
\textbf{wrapper.json}
    \begin{alltt}
    [{"predicates":[{"name":"Q","occurrences":[[0,8]]}], \\
        "timepoint":65536,"timestamp":"1970-01-01 00:00:01"}, \\
    {"predicates":[],"timepoint":65537,"timestamp":"1970-01-01 00:00:02"}]
    \end{alltt}

\textbf{monpoly.log}
    \begin{alltt}
@1 Q(0, 8) ; \\
@2 ;
    \end{alltt}


    
\end{frame}