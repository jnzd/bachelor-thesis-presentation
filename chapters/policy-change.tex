\section{Algorithm and Policy Change}

\begin{frame}
    \frametitle{Policy Change}
    \begin{itemize}
        \item Start a new monitor with the new policy
        \item \textbf{Goal:} Our monitor evaluates the new policy at the current time point just as if it had seen the same trace as the old monitor
        \item \textbf{Naive approach:} Read entire trace again
        \item \textbf{Idea:} Reduce the size of the trace by removing events that do not influence how the new policy gets evaluated
    \end{itemize}
\end{frame}

\begin{frame}{Interval Operators}
    Let $I$ and $J$ be two intervals, then
    \begin{itemize}
        \item $I \oplus J = \{i+j \mid i \in I, j \in J \}$  
        \begin{itemize}
            \item $[0,3] \oplus [-2,4] = [-2,7]$
        \end{itemize}
        \item $I \Cup J$ is the smallest interval that contains all elements that are in at least one of the intervals $I$ and $J$.
        \begin{itemize}
            \item $[-4,1] \Cup [4,5] = [-4,5]$
        \end{itemize}
    \end{itemize}
    
\end{frame}


\begin{frame}
    \frametitle{Relative Intervals}
    \begin{definition} 
        \label{def:rel-int}
        The relative interval of the formula $\phi$, $\RI(\phi) \subseteq \mathbb{Z}$ is defined recursively over the formula structure: 
        $\RI(\phi) =$
        \begin{equation*}
            \begin{cases}
                \{0\}     & \text{atomic formula,} \\ 
                \RI(\psi) & \text{$\neg \psi$, 
                                    $\exists x.\psi$,} \\ &\text{or $\forall x.\psi$,} \\
                \RI(\psi) \Cup \RI(\chi) & \text{$\psi \lor \chi, or
                                                \psi \land \chi$,} \\
                (-b,0] \Cup ((-b,-a] \oplus \RI(\psi)) & \text{$\Previous{[a,b)}\psi$,} \\
                [0,b) \Cup ([a,b) \oplus \RI(\psi)) & \text{$\Next{[a,b)}$,}\\
                (-b,0] \Cup ((-b,0] \oplus \RI(\psi)) \Cup ((-b,-a] \oplus \RI(\chi)) & \text{$\psi \Since{}_{[a,b)} \chi$,} \\
                [0,b) \Cup ([0,b) \oplus \RI(\psi)) \Cup ([a,b) \oplus \RI(\chi)) & \text{$\psi \Until{[a,b)} \chi$,} \\
                % [0,b) \Cup ([0,b) \oplus \RIr(\rho)) & \text{$\Fregex{[a,b)} \rho$, and}\\
                % (-b,0] \Cup ((-b,0] \oplus \RIr(\rho)) & \text{$\Pregex{[a,b)} \rho$.}\\
            \end{cases}
        \end{equation*}
    \end{definition}
    Basin et al. \cite{Basin2016}
\end{frame}

\begin{frame}{Relative Intervals Example}
    \begin{align*}
        \neg (\texttt{loc\_accessed(i, "advertising")}) 
        \Since{[0,30\text{d})}
        \texttt{perm\_revoked(i)}
    \end{align*}
    
\end{frame}

\begin{frame}
    \frametitle{Extended Relative Intervals}

    \begin{definition}
        \label{def:e-rel-int-ops}
        Let $M$ and $N$ be two masked predicate maps and $T$ a positive interval, then 
        \begin{align*}
            M \Cupmerge N = 
                & \{ p(l) \rightarrow (I \Cup J) \mid 
                    p(l) \rightarrow I \in m \text{ and } 
                    p(l) \rightarrow J \in n \} \\
                & \cup \{p(l) \rightarrow I \mid  
                    (p(l) \rightarrow I \in m \text{ and }
                    p(l) \in \keys(M) \setminus \keys(N)) \} \\
                & \cup \{p(l) \rightarrow I \mid  
                    (p(l) \rightarrow I \in n \text{ and }
                    p(l) \in \keys(N) \setminus \keys(M))
                    \} \\
            T \Cupext M = 
                & \{ p(l) \rightarrow (T \Cup I) \mid 
                    p(l) \rightarrow I \in M \} \\
            T \oplusext M = 
                & \{ p(l) \rightarrow (T \oplus I) \mid 
                    p(l) \rightarrow I \in M \} \\
        \end{align*}
    \end{definition}
    
\end{frame}

\begin{frame}
    \frametitle{Extended Relative Intervals}
    \begin{definition}
        \label{def:e-rel-int}
        The extended relative interval of the formula $\varphi$, $\ERI(\varphi)$ is defined recursively over the formula structure:
        $\ERI(\varphi) =$
        \begin{equation*}
            \begin{cases}
                \{\} 
                    & \text{if } \varphi \text{ is an atomic formula} \\
                    & \text{and not a predicate,} \\ 
                \{p(m) \rightarrow [0,0]\} 
                    & \text{if } \varphi \text{ is a predicate with name } \\
                    & p \text{ and mask } m, \\
                \ERI(\psi) 
                    & \text{if } \varphi \text{ is of the form } \neg \psi, \exists x.\psi, \\
                    & \text{or } \forall x.\psi, \\
                \dots
                % \ERI(\psi) \Cupmerge \ERI(\chi) 
                %     & \text{if } \varphi \text{ is of the form } \psi \lor \chi, \\
                %     & \text{or } \psi \land \chi, \\
                % (-b,0] \Cupext ((-b,-a] \oplusext \ERI(\psi)) 
                %     & \text{if } \varphi \text{ is of the form } \Previous{[a,b)} \psi, \\
                % [0,b) \Cupext ([a,b) \oplusext \ERI(\psi)) 
                %     & \text{if } \varphi \text{ is of the form } \Next{[a,b)} \psi,\\
                % (-b,0] \Cupext ((-b,0] \oplusext \ERI(\psi)) \Cupmerge ((-b,-a] \oplusext \ERI(\chi)) 
                %     & \text{if } \varphi \text{ is of the form } \psi \Since{[a,b)} \chi, \\
                % [0,b) \Cupext ([0,b) \oplusext \ERI(\psi)) \Cupmerge ([a,b) \oplusext \ERI(\chi)) 
                %     & \text{if } \varphi \text{ is of the form } \psi \Until{[a,b)} \chi, \\
                % [0,b) \Cupext ([0,b) \oplusext \ERIr(\psi)) 
                %     & \text{if } \varphi \text{ is of the form } \Fregex{[a,b)} \psi, \text{ and}\\
                % (-b,0] \Cupext ((-b,0] \oplusext \ERIr(\psi)) 
                %     & \text{if } \varphi \text{ is of the form } \Pregex{[a,b)} \psi.
            \end{cases}
        \end{equation*}
    \end{definition}
    
\end{frame}


\begin{frame}{Extended Relative Intervals Example}
    
\end{frame}