\section{Background and Motivation}


\begin{frame}{Agenda}
    \tableofcontents
\end{frame}

\subsection{Monitors and Databases}


\subsection{MonPoly}

\begin{frame}
    \frametitle{MonPoly}
    \begin{itemize}
        \item Runtime Monitor
        \item Metric First Order Temporal Logic (MFOTL)
        \item MFOTL temporal operators (left) and syntactic sugar (right)
        \begin{columns}
            \begin{column}{0.4\textwidth}
                \begin{itemize}
                    \item $\Next{I}$ ("Next")
                    \item $\Previous{I}$ ("Previous")
                    \item $\Until{I}$ ("Until")
                    \item $\Since{I}$ ("Since")
                \end{itemize}
            \end{column}
            \begin{column}{0.4\textwidth}
                \begin{itemize}
                    \item $\Always{I}$ ("Always")
                    \item $\Historically{I}$ ("Historically")
                    \item $\Eventually{I}$ ("Eventually")
                    \item $\Once{I}$ ("Once")
                \end{itemize}
            \end{column}
        \end{columns}
        \vspace{.2cm}
        \textit{The interval $I$ restricts the time points relevant to an operator.}
        \item Time Point vs. Time Stamp
        \begin{itemize}
            \item \textbf{Time Point}: Time stamped collection of events (predicates), often referred to by an integer index
            \item \textbf{Time Stamp}: Specifies a moment in time, e.g. seconds since Unix epoch
        \end{itemize}

    \end{itemize}
\end{frame}

\begin{frame}{Online and Offline Monitoring}
    \begin{itemize}
        \item \textbf{Online Monitoring} is done while the system is running
        \item \textbf{Offline Monitoring} is done on a log after a system has completed
        \item We focus on the Online Monitoring problem
    \end{itemize}
    
\end{frame}


\subsection{Time-Series Databases (QuestDB)}


\begin{frame}{Time-Series Databases}
    \begin{itemize}
        \item A time-series database is optimized for the insertion and retrieval of temporal data.
        \item QuestDB
        \begin{itemize}
            \item Largely compatible with PostgreSQL
            \item Adds a {\color{red} \texttt{TIMESTAMP}} datatype in addition to the usual types \texttt{INT}, \texttt{STRING}, etc.
        \end{itemize}
        \item Dedicated time stamp column (analogous to an index column)
    \end{itemize}
    \vspace{0.6cm}
    \centering
    \includegraphics[width=0.5\linewidth]{diagrams/questdb.png}
    
\end{frame}

\begin{frame}{Monitor vs. Database Approach}
    \begin{columns}
        \begin{column}{0.5\textwidth}
            {\Large Monitor (MonPoly)}
            \begin{itemize}
                \item Good when: 
                    \begin{itemize}
                        \item Data changes often
                        \item Policy changes infrequently
                    \end{itemize}
                \item Bad when:
                    \begin{itemize}
                        \item Policy changes frequently
                    \end{itemize}
            \end{itemize}

        \end{column}
        \begin{column}{0.5\textwidth}
            {\Large Database (QuestDB)}
            \begin{itemize}
                \item Good when:
                    \begin{itemize}
                        \item Data stays almost constant
                        \item Varying, frequent queries
                    \end{itemize}

                \item Bad when:
                    \begin{itemize}
                        \item Data changes rapidly
                    \end{itemize}
            \end{itemize}


        \end{column}
    \end{columns}
    
\end{frame}


\begin{frame}{Motivation}
    \begin{itemize}
        \item Exploit the combination of monitoring and time-series databases
        \begin{itemize}
            \item Allows us to offer a policy change method
        \end{itemize}
        \item More accessible interface
        \begin{itemize}
            \item Web services
            \item Distributed systems
        \end{itemize}
        \item Increase fault tolerance
    \end{itemize}
    
\end{frame}